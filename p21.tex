\section{Pregunta N$^{\circ}$21b\qquad Carlos Daniel Malvaceda Canales}

\begin{frame}
	\begin{enumerate}\setcounter{enumi}{20}
		\item

		      Sean $a=0.000063381158, b=73.688329$ y $c=-73.687711$.
		      Determine la aritmética de punto flotante para:

		      \begin{enumerate}[b)]
			      \item

			            \begin{math}
				            \left(a+b\right)+c
			            \end{math}.
		      \end{enumerate}
	\end{enumerate}

	\begin{solution}
		\begin{enumerate}[b)]
			\item
			      Para realizar la aritmética de punto flotante de (a + b) + c, primero se deben normalizar , se considerando :
			      \begin{math}
				      \mathbb{F}\left(10,6,-5,5\right)
			      \end{math}

			      \begin{align*}
				      a = 0.000063381158 = 0.63381158 \times 10^{-4} \\
				      b = 73.688329 = 0.73688329 \times 10^{2}       \\
				      c = -73.687711 = -0.73687711 \times 10^{2}
			      \end{align*}
			      \begin{math}
				      \operatorname{fl}\left(a\right)=
				      0.633812\times 10^{-4}
			      \end{math},
			      \begin{math}
				      \operatorname{fl}\left(b\right)=
				      0.736883\times 10^{2}
			      \end{math},
			      \begin{math}
				      \operatorname{fl}\left(c\right)=
				      -0.736877\times 10^{2}
			      \end{math}
			      \\Sumando a y b :
			      \begin{align*}
				      \operatorname{fl}\left(\operatorname{fl}\left(a\right)+\operatorname{fl}\left(b\right)\right)
				       & =
				      0.736884\times 10^{2} \\
				      \operatorname{fl}\left(\operatorname{fl}\left(a\right)+\operatorname{fl}\left(b\right)\right)+
				      \operatorname{fl}\left(c\right)
				       & =0.7\times 10^{-3} \\
				      \operatorname{fl}\left(\operatorname{fl}\left(\operatorname{fl}\left(a\right)+\operatorname{fl}\left(b\right)\right)+
				      \operatorname{fl}\left(c\right)\right)
				       & =
				      0.7\times 10^{-3}
			      \end{align*}

			      \end{solution}


			      % Para realizar la aritmética de punto flotante de
			      % \begin{math}
			      % 	\left(a+b\right)+c
			      % \end{math},
			      % primero se deben normalizar, considerando:
			      % \begin{math}
			      % 	\mathbb{F}\left(10,6,-5,5\right)
			      % \end{math}.

			      % \begin{align*}
			      % 	a & =0.000063381158=0.63381158 \times 10^{-4} \\
			      % 	b & =73.688329=0.73688329\times 10^{2}        \\
			      % 	c & =-73.687711=-0.73687711\times 10^{2}
			      % \end{align*}

			      % \begin{math}
			      % 	\operatorname{fl}\left(a\right)=
			      % 	0.633812\times 10^{-4}
			      % \end{math},
			      % \begin{math}
			      % 	\operatorname{fl}\left(b\right)=
			      % 	0.736883\times 10^{2}
			      % \end{math},
			      % \begin{math}
			      % 	\operatorname{fl}\left(c\right)=
			      % 	-0.736877\times 10^{2}
			      % \end{math}
			      % \\Sumando a y b :
			      % \begin{align*}
			      % 	a + b = 0.63381158 \times 10^{-4} +  0.73688329 \times 10^{2}
			      % \end{align*}
			      % Para sumar estos números, primero se ajustan los exponentes.
		\end{enumerate}
	\end{solution}
\end{frame}
