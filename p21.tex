\section{Pregunta N$^{\circ}$21b\qquad Carlos Daniel Malvaceda Canales}

\begin{frame}
	\begin{enumerate}\setcounter{enumi}{20}
		\item

		      Sean $a=0.000063381158, b=73.688329$ y $c=-73.687711$.
		      Determine la aritmética de punto flotante para:

		      \begin{enumerate}[b)]
			      \item

			            \begin{math}
				            \left(a+b\right)+c
			            \end{math}.
		      \end{enumerate}
	\end{enumerate}

    \begin{solution}
        Para realizar la aritmética de punto flotante de (a + b) + c, primero se deben normalizar 
        \begin{align*}
            a = 0.000063381158 = 0.63381158 \times 10^{-4}
            b = 73.688329 = 0.73688329 \times 10^{2} 
            c = -73.687711 = -0.73687711 \times 10^{2}
        \end{align*}
        Sumando a y b :
        \begin{align*}
            a + b = 0.63381158 \times 10^{-4} +  0.73688329 \times 10^{2}
        \end{align*} 
        Para sumar estos números, primero se ajustan los exponentes.
        \begin{math}
        
        \end{math}
    \end{solution}
        

\end{frame}
