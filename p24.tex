\begin{frame}
	\begin{enumerate}\setcounter{enumi}{23}
		\item

		      Si tenemos $\beta=2$, $t=3$, $L=-2$ y $U=2$, determine los
		      números de máquina que contiene dicho intervalo y además
		      determine

		      \begin{enumerate}[b)]
			      \item

			            \begin{math}
				            \dfrac{24}{32}\oplus
				            \dfrac{7}{32}
			            \end{math}.
		      \end{enumerate}
	\end{enumerate}

	\begin{solution}
		\begin{definition}[Conjunto de números de punto flotante~\cite{Kincaid1994}]
			Sea el \alert{conjunto de números de punto flotante}
			\begin{equation*}
				\mathbb{F}\left(\beta,t,L,U\right)=
				\left\{0\right\}+
				\left\{
				x\in\mathbb{R}\mid
				x=
				\left(-1\right)^{s}
				\beta^{e}
				\sum_{i=1}^{t}a_{i}\beta^{-i}
				\right\}
			\end{equation*}
			con $t$ dígitos significativos, base $\beta\geq2$,
			$0\leq a_{i}\leq\beta-1$ y rango $\left(L,U\right)$ con
			$0<L\leq e\leq U>0$.
		\end{definition}

		El
		\begin{math}
			\operatorname{card}\mathbb{F}\left(\beta=2,t=3,L=-2,U=2\right)=
			2\left(\beta-1\right)\beta^{t-1}\left(U-L+1\right)+1=
			2\left(2-1\right)2^{3-1}\left(2-\left(-2\right)+1\right)+1=
			41
		\end{math}.
	\end{solution}
\end{frame}

% Dado un $x\in\mathbb{R}$ en la notación posicional normalizada
% \begin{equation*}
% 	.
% \end{equation*}