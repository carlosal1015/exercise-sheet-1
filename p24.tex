\section{Pregunta N$^{\circ}$24a\qquad Breiner Catalino Morales}

\begin{frame}
	\begin{enumerate}\setcounter{enumi}{23}
		\item

		      Si tenemos $\beta=2$, $t=3$, $L=-2$ y $U=2$, determine los
		      números de máquina que contiene dicho intervalo y además
		      determine

		      \begin{enumerate}[b)]
			      \item

			            \begin{math}
				            \dfrac{24}{32}\oplus
				            \dfrac{7}{32}
			            \end{math}.
		      \end{enumerate}
	\end{enumerate}

	\begin{solution}
		\begin{definition}[Conjunto de números de punto flotante~\cite{Kincaid1994}]
			Sea el \alert{conjunto de números de punto flotante}
			\begin{equation*}
				\mathbb{F}\left(\beta,t,L,U\right)=
				\left\{0\right\}+
				\left\{
				x\in\mathbb{R}\mid
				x=
				\left(-1\right)^{s}
				\beta^{e}
				\sum_{i=1}^{t}a_{i}\beta^{-i}
				\right\}
			\end{equation*}
			con $t$ dígitos significativos, base $\beta\geq2$,
			$0\leq a_{i}\leq\beta-1$ y rango $\left(L,U\right)$ con
			$0<L\leq e\leq U>0$.
		\end{definition}
	\end{solution}
\end{frame}

\begin{frame}
	\begin{solution}
		Si $x\in\mathbb{F}\left(\beta,t,L,U\right)$, entonces
		% $-x\in\mathbb{F}\left(\beta,t,L,U\right)$ y
		\begin{math}
			x_{\min}=
			\beta^{L-1}\leq
			\left|x\right|\leq
			\beta^{U}\left(1-\beta^{-t}\right)=
			x_{\max}
		\end{math}.
		Para
		\begin{math}
			\mathbb{F}
			\left(
			\beta=2,
			t=3,
			L=-2,
			U=2
			\right)
		\end{math}:
		\begin{equation*}
			\alert{\dfrac{1}{8}}=
			2^{-2-1}=
			x_{\min}\leq
			\left|x\right|\leq
			x_{\max}=
			2^{2}\left(1-2^{-3}\right)=
			4\left(1-\dfrac{1}{8}\right)=
			\alert{\dfrac{7}{2}}.
		\end{equation*}
		\alert{El total de número de máquinas} están dados por los que aparecen en la siguiente tabla,
		sus opuestos y el cero, en total el
		\begin{math}
			\operatorname{card}
			\mathbb{F}
			\left(
			\beta=2,
			t=3,
			L=-2,
			U=2
			\right)=
			2\left(\beta-1\right)
			{\beta}^{t-1}
			\left(U-L+1\right)+1=
			2\left(2-1\right)
			{2}^{3-1}
			\left(
			2-\left(-2\right)+1
			\right)+
			1=
			\alert{41}
		\end{math}.
		% Los números reales $x$ se encuentran en un determinado intervalo
		\begin{table}[ht!]
			\renewcommand{\arraystretch}{2.5}
			\begin{tabular}{|>{$}c<{$}|>{$}c<{$}|>{$}c<{$}|>{$}c<{$}|>{$}c<{$}|>{$}c<{$}|}
				\hline
				-2
				 & -1
				 & 0
				 & 1
				 & 2
				\\
				\hline
				{\left(0.100\right)}_{2}\times 2^{-2}=\alert{\dfrac{1}{8}}
				 & {\left(0.100\right)}_{2}\times 2^{-1}=\dfrac{1}{4}
				 & {\left(0.100\right)}_{2}\times 2^{0}=\dfrac{1}{2}
				 & {\left(0.100\right)}_{2}\times 2^{1}=\dfrac{1}{1}
				 & {\left(0.100\right)}_{2}\times 2^{2}=\dfrac{2}{1}
				\\
				\hline
				{\left(0.101\right)}_{2}\times 2^{-2}=\dfrac{5}{32}
				 & {\left(0.101\right)}_{2}\times 2^{-1}=\dfrac{5}{16}
				 & {\left(0.101\right)}_{2}\times 2^{0}=\dfrac{5}{8}
				 & {\left(0.101\right)}_{2}\times 2^{1}=\dfrac{5}{4}
				 & {\left(0.101\right)}_{2}\times 2^{2}=\dfrac{5}{2}
				\\
				\hline
				{\left(0.110\right)}_{2}\times 2^{-2}=\dfrac{3}{16}
				 & {\left(0.110\right)}_{2}\times 2^{-1}=\dfrac{3}{8}
				 & {\left(0.110\right)}_{2}\times 2^{0}=\dfrac{3}{4}
				 & {\left(0.110\right)}_{2}\times 2^{1}=\dfrac{3}{2}
				 & {\left(0.110\right)}_{2}\times 2^{2}=\dfrac{3}{1}
				\\
				\hline
				{\left(0.111\right)}_{2}\times 2^{-2}=\dfrac{7}{32}
				 & {\left(0.111\right)}_{2}\times 2^{-1}=\dfrac{7}{16}
				 & {\left(0.111\right)}_{2}\times 2^{0}=\dfrac{7}{8}
				 & {\left(0.111\right)}_{2}\times 2^{1}=\dfrac{7}{4}
				 & {\left(0.111\right)}_{2}\times 2^{2}=\alert{\dfrac{7}{2}}
				\\
				\hline
			\end{tabular}
		\end{table}
	\end{solution}
\end{frame}

\begin{frame}
	\begin{align*}
		\dfrac{24}{32}\oplus
		\dfrac{7}{32}
		 & =
		\operatorname{fl}
		\left(
		\operatorname{fl}
		\left(
			\dfrac{24}{32}
			\right)+
		\operatorname{fl}
		\left(
			\dfrac{7}{32}
			\right)
		\right) \\
		 & =
		\operatorname{fl}
		\left(
		\dfrac{24}{32}
		+
		\dfrac{7}{32}
		\right) \\
		 & =
		\operatorname{fl}
		\left(
		\dfrac{31}{32}
		\right) \\
		 & =1.
	\end{align*}
\end{frame}