\section{Pregunta N$^{\circ}$2\qquad Aldo Luna Bueno}

\begin{frame}

	\begin{enumerate}\setcounter{enumi}{1}
		\item

		      Si $x=0.43257143$ e $y=0.43257824$.

		      \begin{enumerate}[a)]
			      \item\label{q:2.a}

			      Use aritmética de redondeo a cinco cifras para calcular
			      \begin{math}
				      \operatorname{fl}\left(x\right)
			      \end{math}
			      y
			      \begin{math}
				      \operatorname{fl}\left(y\right)
			      \end{math}.

			      \item\label{q:2.b}

			      Calcular los errores relativos y absolutos.

			      \item\label{q:2.c}

			      Calcular $x+y$, $x-y$, $xy$ y $\frac{x}{y}$.

			      \item\label{q:2.d}

			      Calcular los errores relativos para las operaciones
			      realizadas en el inciso anterior.
		      \end{enumerate}
	\end{enumerate}

	\begin{solution}
		\begin{definition}[Redondeo]
			Dado un $x\in\mathbb{R}$ en la notación posicional normalizada
			\begin{equation*}
				\operatorname{fl}\left(x\right)=
				{\left(-1\right)}^{s}
				\left(
				0.a_{1}\dotsc,\widetilde{a}_{t}
				\right)\cdot
				\beta^{e},\quad
				\widetilde{a}_{t}=
				\begin{cases}
					a_{t},   & \text{si } a_{t+1}\leq\frac{\beta}{2}, \\
					a_{t}+1, & \text{si } a_{t+1}\geq\frac{\beta}{2}.
				\end{cases}
			\end{equation*}
		\end{definition}
	\end{solution}
\end{frame}

\begin{frame}
	\begin{solution}
		\begin{enumerate}[a)]
			\item

			      .

			\item

			      .

			\item

			      \begin{math}
				      x+y=
				      \operatorname{fl}\left(x+y\right).
			      \end{math}

			\item

			      .
		\end{enumerate}
	\end{solution}
\end{frame}