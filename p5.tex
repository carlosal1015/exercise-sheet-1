\section{Pregunta N$^{\circ}$5\qquad Khalid Zaid Izquierdo Ayllón}

\begin{frame}
	\begin{enumerate}\setcounter{enumi}{4}
		\item

		      Determine el valor decimal, la suma y la diferencia de los
		      números binarios $A=11100111$ y $B=10111111$, suponiendo
		      que:

		      \begin{multicols}{2}

			      \begin{enumerate}[a)]
				      \item

				            Ambos están representados en magnitud y signo.

				      \item

				            Ambos están representados en complemento a 2.
			      \end{enumerate}
		      \end{multicols}
	\end{enumerate}

	\begin{solution}

		\begin{table}[ht!]
			\begin{math}
				A=
			\end{math}
			\begin{tabular}{|>{$}c<{$}|>{$}c<{$}|>{$}c<{$}|>{$}c<{$}|>{$}c<{$}|>{$}c<{$}|>{$}c<{$}|>{$}c<{$}|}
				\hline
				\alert{1} & 1 & 1 & 0 & 0 & 1 & 1 & 1 \\
				\hline
			\end{tabular}\qquad\qquad\qquad
			\begin{math}
				B=
			\end{math}
			\begin{tabular}{|>{$}c<{$}|>{$}c<{$}|>{$}c<{$}|>{$}c<{$}|>{$}c<{$}|>{$}c<{$}|>{$}c<{$}|>{$}c<{$}|}
				\hline
				\alert{1} & 0 & 1 & 1 & 1 & 1 & 1 & 1 \\
				\hline
			\end{tabular}
		\end{table}

		\begin{enumerate}[a)]
			\item

			      En la representación magnitud y signo, el primer bit
			      significativo es reservado para el signo
			      ($0$ para los positivos y $1$ para los negativos).
			      Así, los números $A$ y $B$ son negativos.
			      Ahora, analizamos el resto de cifras para obtener el
			      número decimal, esto se realizará con la conversión de
			      base $2$ a base $10$.

			      Convertiremos el número ${1100111}_{\left(2\right)}$ a
			      base $10$.

			      \begin{equation*}
				      1\times 2^{0}+
				      1\times2^{1}+
				      1\times2^{2}+
				      0\times2^{3}+
				      0\times2^{4}+
				      1\times2^{5}+
				      1\times2^{6}=
				      103.
			      \end{equation*}

			      Su primer bit significativo es $1$, es decir, el número
			      es negativo, este resultado se multiplicará por $-1$.

			      Dándonos así que el valor en base $10$ del binario $A$ es
			      $-103$.
			      $A=-{103}_{\left(10\right)}$.

			      De forma análoga para $B$, su valor en base $10$ es $-63$.
			      $B=-{63}_{\left(10\right)}$.
		\end{enumerate}
	\end{solution}
\end{frame}

\begin{frame}
	\begin{solution}
		\begin{enumerate}[a)]
			\item

			      Para calcular la suma de $A$ y $B$, primero
			      pondremos uno encima del otro de la siguiente forma:

			      \begin{table}[ht!]
				      \begin{tabular}{>{$}c<{$} >{$}r<{$} >{$}l<{$} >{$}c<{$}}
					        & \alert{1} & 1100111 & \left(-103\right) \\
					      + & \alert{1} & 0111111 & \left(-63\right)  \\
					      \hline
				      \end{tabular}
			      \end{table}

			      Como se puede apreciar, el primer bit significativo está
			      apartado, dado que este no se usará para la suma de estos
			      números en base $2$, además, a la derecha se puede
			      apreciar dichos números en base $10$ para corroborar al
			      finalizar la suma.
			      En este caso que las primeras cifras significativas de
			      cada número son iguales se mantendrá esta cifra al
			      finalizar la suma, dado que al ser iguales el signo no
			      cambiará.

			      Realizaremos la suma de las cifras de la magnitud en un
			      espacio aparte de la forma tradicional resultando de esta
			      forma:

			      \begin{table}[ht!]
				      \begin{tabular}{>{$}c<{$} >{$}c<{$}}
					        & 1100111  \\
					      + & 0111111  \\
					      \hline
					        & 10100110
				      \end{tabular}
			      \end{table}

			      Podemos darnos cuenta que, al sumar $A$ y $B$ usando solo
			      las cifras de la magnitud y siendo estas $7$, el
			      resultado nos da $8$ cifras, esto quiere decir que
			      obtuvimos un Carry Out de valor $1$, lo cual nos da a
			      entender que ocurrió un Overflow.

			      Dicha resultado tiene sentido dado que la suma de sus
			      números en base $10$ nos da $-166$ cuyo número no es
			      posible representar en magnitud y signo para $8$ bits,
			      dado que el menor número posible es $-127$, por ende no
			      es posible realizar la suma de estos números para $8$
			      bits.

			      \begin{table}[ht!]
				      \begin{tabular}{>{$}c<{$} >{$}r<{$} >{$}l<{$} >{$}c<{$}}
					        & \alert{1} & 1100111 & \left(-103\right) \\
					      + & \alert{1} & 0111111 & \left(-63\right)  \\
					      \hline
					        & \alert{1} & 0100110 & \left(-166\right)
				      \end{tabular}
			      \end{table}

			      Al devolver el número sin el Carry Out a la suma
			      original, nos podemos dar cuenta que si convertimos el
			      resultado a base $10$, no nos dará $-166$, demostrando
			      una vez más que no es posible la suma.
		\end{enumerate}
	\end{solution}
\end{frame}


\begin{frame}
	\begin{solution}
		\begin{enumerate}[a)]
			\item

			      Para calcular la resta de $A$ y $B$, seguiremos un
			      proceso similar al anterior, dado que $A-B$ es lo mismo
			      que decir $A+\left(-B\right)$, lo único diferente que
			      realizaremos será cambiarle el signo al número binario
			      $B$, para ello solo tendremos que cambiar el $1$ de su
			      primera cifra significativa por un $0$ y resolveremos la
			      suma como el caso anterior.

			      \begin{table}[ht!]
				      \begin{tabular}{>{$}c<{$} >{$}r<{$} >{$}l<{$} >{$}c<{$}}
					        & \alert{1} & 1100111 & \left(-103\right) \\
					      - & \alert{1} & 0111111 & \left(-63\right)  \\
					      \hline
				      \end{tabular}
				      \qquad
				      \begin{math}
					      \Longrightarrow
				      \end{math}
				      \qquad
				      \begin{tabular}{>{$}c<{$} >{$}r<{$} >{$}l<{$} >{$}c<{$}}
					        & \alert{1} & 1100111 & \left(-103\right) \\
					      + & \alert{0} & 0111111 & \left(63\right)   \\
					      \hline
				      \end{tabular}
			      \end{table}

			      Realizaremos la resta de las cifras de la magnitud, dado
			      que las cifras del primer bit significativo son
			      distintas, para ello pondremos abajo la magnitud menor,
			      es decir sin considerar el signo, y arriba la mayor,
			      obteniendo así el siguiente resultado:

			      \begin{table}[ht!]
				      \begin{tabular}{>{$}c<{$} >{$}c<{$}}
					        & 1100111 \\
					      - & 0111111 \\
					      \hline
					        & 0101000
				      \end{tabular}
			      \end{table}

			      Una vez realizada esa operación, la colocamos en nuestra
			      suma principal, además, pondremos el signo de la magnitud
			      más grande en el primer bit significativo, dándonos así
			      lo siguiente:

			      \begin{table}[ht!]
				      \begin{tabular}{>{$}c<{$} >{$}r<{$} >{$}l<{$} >{$}c<{$}}
					        & \alert{1} & 1100111 & \left(-103\right) \\
					      + & \alert{0} & 0111111 & \left(63\right)   \\
					      \hline
					        & \alert{1} & 0101000 & \left(-40\right)  \\
				      \end{tabular}
			      \end{table}

			      De la resta de $A$ y $B$ obtendremos el número binario
			      $10101000$ el cual al transformarlo a base decimal sería
			      $-40$, el mismo resultado que nos daría la operación de
			      haberla realizado en base $10$.
		\end{enumerate}
	\end{solution}
\end{frame}

\begin{frame}
	\begin{solution}
		\begin{enumerate}[b)]
			\item
			      En la representación complemento a $2$, la conversión a
			      decimal se hace similar a una conversión en números
			      naturales, salvo que la primera cifra significativa se
			      restará en vez de sumarse como las demás.

			      Convertiremos el número $11100111_{\left(2\right)}$ a base $10$.

			      \begin{equation*}
				      1\times 2^{0}+
				      1\times2^{1}+
				      1\times2^{2}+
				      0\times2^{3}+
				      0\times2^{4}+
				      1\times2^{5}+
				      1\times2^{6}-
				      1\times2^{7}=
				      -25
			      \end{equation*}

			      Dándonos así que el valor en base $10$ del binario $A$ es
			      $-25$.
			      $A=-{25}_{\left(10\right)}$.

			      De forma análoga para $B$, su valor en base $10$ es $-65$.
			      $B=-{65}_{\left(10\right)}$.

			      Para calcular la suma de $A$ y $B$ se realizará como una
			      suma normal de números en mismas bases, obteniendo el
			      siguiente resultado:

			      \begin{table}[ht!]
				      \begin{tabular}{c c c}
					        & 11100111 & (-25) \\
					      + & 10111111 & (-65) \\
					      \hline
					        & 10100110 & (-90)
				      \end{tabular}
			      \end{table}
			      Con un Carry out=1, no ocurre overflow, ya que la primera
			      cifra significativa del resultado, ignorando el Carry
			      out, es igual a las otras $2$, además, convirtiendo el
			      resultado a base decimal nos resulta $-90$, tal cual como
			      resultaría la suma en base $10$.
		\end{enumerate}
	\end{solution}
\end{frame}

\begin{frame}
	\begin{solution}
		\begin{enumerate}[b)]
			\item

			      Para la resta de $A$ y $B$ se realizará como en el
			      apartado $A$, es decir realizaremos $A+\left(-B\right)$,
			      transformaremos el binario $B$ a $-B$, para ello, una de
			      las formas de hacerlo es buscar de derecha a izquierda la
			      primera cifra $1$, y cambiar las cifras siguientes a
			      esta, de forma que los $1$ se vuelvan $0$ y viceversa.
			      Por último, se suman las cifras como una suma normal en
			      la misma base.

			      \begin{table}[ht!]
				      \begin{tabular}{c c c}
					        & 11100111 & (-25) \\
					      - & 10111111 & (-65) \\
					      \hline
				      \end{tabular}
				      \qquad
				      \begin{math}
					      \Longleftarrow
				      \end{math}
				      \qquad
				      \begin{tabular}{c c c}
					        & 11100111 & (-25) \\
					      + & 01000001 & (65)  \\
					      \hline
					        & 00101000 & (40)  \\
				      \end{tabular}
			      \end{table}
			      De esta forma obtendremos la resta de $A$ y $B$, en la
			      cual tenemos un Carry out=1, el cual ignoraremos ya que
			      la representación es con $8$ bits y transformando el
			      resultado a base decimal nos resulta $40$, el cual sería
			      el resultado en base $10$.
		\end{enumerate}
	\end{solution}
\end{frame}