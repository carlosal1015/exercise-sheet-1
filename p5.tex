\section{Pregunta N$^{\circ}$5\qquad Izquierdo Ayllón Khalid Zaid}

\begin{frame}
	\begin{enumerate}\setcounter{enumi}{4}
		\item

		      Determine el valor decimal, la suma y la diferencia de los
		      números binarios $A=11100111$ y $B=10111111$, suponiendo
		      que:

		      \begin{multicols}{2}

			      \begin{enumerate}[a)]
				      \item

				            Ambos están representados en magnitud y signo.

				      \item

				            Ambos están representados en complemento a 2.
			      \end{enumerate}
		      \end{multicols}
	\end{enumerate}

	\begin{solution}

		\begin{table}[ht!]
			\begin{math}
				A=
			\end{math}
			\begin{tabular}{|>{$}c<{$}|>{$}c<{$}|>{$}c<{$}|>{$}c<{$}|>{$}c<{$}|>{$}c<{$}|>{$}c<{$}|>{$}c<{$}|}
				\hline
				\alert{1} & 1 & 1 & 0 & 0 & 1 & 1 & 1 \\
				\hline
			\end{tabular}\qquad\qquad\qquad
			\begin{math}
				B=
			\end{math}
			\begin{tabular}{|>{$}c<{$}|>{$}c<{$}|>{$}c<{$}|>{$}c<{$}|>{$}c<{$}|>{$}c<{$}|>{$}c<{$}|>{$}c<{$}|}
				\hline
				\alert{1} & 0 & 1 & 1 & 1 & 1 & 1 & 1 \\
				\hline
			\end{tabular}
		\end{table}

		\begin{enumerate}[a)]
			\item
                    Dado que están representados en magnitud y signo, el primer bit significativo es reservado exclusivamente para el símbolo, además sabiendo que el 0 se usa para los positivos y el 1, para los negativos, tenemos que, tanto el número de A y el número de B son negativos.
                    Una vez identificado el signo, analizaremos el resto de cifras para obtener el número decimal, esto se realizará de la forma cotidiana de conversión de base n a base 10. Tenemos:

                    Convertiremos el número $1100111_{(2)}$ a base 10.

                    $1\times2^0+1\times2^1+1\times2^2+0\times2^3+0\times2^4+1\times2^5+1\times2^6=103$

                    Como su primer bit significativo es 1, lo cual significa que el número es negativo, este resultado se multiplicará por -1.

                    Dándonos así que el valor en base 10 del binario A es -103.

                    $A=-103_{(10)}$

                    Haciendo el proceso de forma análoga para B, nos resulta que el valor en base 10 es -63.

                    $B=-63_{(10)}$

                    Ahora, para calcular la suma de A y B, primero pondremos uno encima del otro de la siguiente forma:

                    \begin{tabular}{c r l c}
                        & \alert{1} & 1100111 & (-103)\\
                       + & \alert{1} & 0111111 & (-63)
                       \hline
                    \end{tabular}

                    Como se puede apreciar, 

			\item


		\end{enumerate}
	\end{solution}
\end{frame}