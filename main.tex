%! Aldo Luna Bueno
%! Alejandro Maycoll Escobar Mejia
%! Carlos Alonso Aznarán Laos
%! Khalid Zaid Izquierdo Ayllon
%! Brian Alberto Huaman Garcia
%! Carlos Daniel Malvaceda Canales
%! Bruno Cipriano Arroyo
%! Breiner Catalino Morales
%! Universidad Nacional de Ingeniería
%! Facultad de Ciencias
%! Lima, Perú
%! Uso:
%! $ sudo pacman -Syu texlive-most zathura # dependencias, visor
%! $ arara solution_primera
%! $ zathura solution_primera.pdf
%! Ver https://wiki.archlinux.org/title/TeX_Live
% arara: clean: {
% arara: --> extensions:
% arara: --> ['aux','log','nav',
% arara: --> 'out','snm','toc','pytxcode','pdf']
% arara: --> }
% arara: lualatex: {
% arara: --> shell: yes,
% arara: --> draft: yes,
% arara: --> interaction: batchmode
% arara: --> }
% arara: biber
% arara: lualatex: {
% arara: --> shell: yes,
% arara: --> draft: yes,
% arara: --> interaction: batchmode
% arara: --> }
% arara: lualatex: {
% arara: --> shell: yes,
% arara: --> synctex: yes,
% arara: --> interaction: batchmode
% arara: --> }
% arara: clean: {
% arara: --> extensions:
% arara: --> ['aux','log','nav',
% arara: --> 'out','snm','toc','pytxcode']
% arara: --> }
\input{preamble}
\begin{document}

\frame{\titlepage}

\begin{frame}
	\frametitle{Lista de N$^{\circ}$ de pregunta / estudiante}
	\tableofcontents
\end{frame}

\section{Pregunta N$^{\circ}$2\qquad Aldo Luna Bueno}
\section{Pregunta N$^{\circ}$5\qquad Alejandro Escobar Mejia}
\section{Pregunta N$^{\circ}$11\qquad Carlos Aznarán Laos}
\section{Pregunta N$^{\circ}$14\qquad Brian Huaman Garcia}
\section{Pregunta N$^{\circ}$16a\qquad Khalid Izquierdo Ayllon}
\section{Pregunta N$^{\circ}$17b\qquad Carlos Malvaceda Canales}
\section{Pregunta N$^{\circ}$19b\qquad Bruno Cipriano Arroyo}
\section{Pregunta N$^{\circ}$21b\qquad Breiner Catalino Morales}
\section{Pregunta N$^{\circ}$24a\qquad Breiner Catalino Morales}

\section{Pregunta N$^{\circ}$2\qquad Bruno Cipriano Arroyo}

\begin{frame}
	\begin{definition}[Redondeo]
		Dado un $x\in\mathbb{R}$ en la notación posicional normalizada,
		definimos la función de redondeo a $t$ dígitos como
		\begin{equation*}
			\operatorname{fl}\left(x\right)=
			{\left(-1\right)}^{s}
			\left(
			0.a_{1}\dotsc,\widetilde{a}_{t}
			\right)\cdot
			\beta^{e},\quad
			\widetilde{a}_{t}=
			\begin{cases}
				\begin{aligned}
					a_{t},
					 & \text{ si }a_{t+1}\leq\frac{\beta}{2}, \\[\baselineskip]
					a_{t}+1,
					 & \text{ si }a_{t+1}\geq\frac{\beta}{2}.
				\end{aligned}
			\end{cases}
		\end{equation*}
	\end{definition}

	\begin{definition}[Error absoluto y el error relativo]
		Para cualquier $x\in\mathbb{R}$, definimos
		\begin{equation*}
			\operatorname{EA}\left(x\right)=
			\left|
			x-
			\operatorname{fl}\left(x\right)\right|,\qquad
			\operatorname{ER}\left(x\right)=
			\frac{
				\left|
				x-
				\operatorname{fl}\left(x\right)\right|
			}{
				\left|
				x
				\right|
			}.
		\end{equation*}
	\end{definition}

	\begin{definition}[Operaciones aritméticas en punto flotante]
		Para cualquier $x,y\in\mathbb{R}$ definimos las operaciones de
		suma, resta, producto y división en $\mathbb{F}$ como
		\begin{equation*}
			\begin{aligned}[c]
				x+y & \coloneqq
				\operatorname{fl}
				\left(
				\operatorname{fl}\left(x\right)+
				\operatorname{fl}\left(y\right)
				\right),        \\
				x-y & \coloneqq
				\operatorname{fl}
				\left(
				\operatorname{fl}\left(x\right)-
				\operatorname{fl}\left(y\right)
				\right),
			\end{aligned}
			\qquad\qquad
			\begin{aligned}[c]
				x\times y   & \coloneqq
				\operatorname{fl}
				\left(
				\operatorname{fl}\left(x\right)\times
				\operatorname{fl}\left(y\right)
				\right),                \\
				\frac{x}{y} & \coloneqq
				\operatorname{fl}
				\left(
				\frac{
					\operatorname{fl}\left(x\right)
				}{
					\operatorname{fl}\left(y\right)
				}
				\right).
			\end{aligned}
		\end{equation*}
	\end{definition}
\end{frame}

\begin{frame}

	\begin{enumerate}\setcounter{enumi}{1}
		\item

		      Si $x=0.43257143$ e $y=0.43257824$.
		      \begin{multicols}{2}
			      \begin{enumerate}[a)]
				      \item\label{q:2.a}

				      Calcule
				      \begin{math}
					      \operatorname{fl}\left(x\right)
				      \end{math}
				      y
				      \begin{math}
					      \operatorname{fl}\left(y\right)
				      \end{math}
				      con redondeo a $5$ cifras.

				      \item\label{q:2.b}

				      Calcule sus errores relativos y absolutos.

				      \item\label{q:2.c}

				      Calcule $x+y$, $x-y$, $x\times y$ y $\dfrac{x}{y}$.

				      \item\label{q:2.d}

				      Calcule los errores relativos de~\eqref{q:2.c}.
			      \end{enumerate}
		      \end{multicols}
	\end{enumerate}
	\begin{solution}
		\begin{enumerate}[a)]
			\item

			      \begin{multicols}{2}
				      \begin{itemize}
					      \item

					            \begin{math}
						            \operatorname{fl}\left(\alert{x}\right)=
					            \end{math}

					      \item

					            \begin{math}
						            \operatorname{fl}\left(\alert{y}\right)=
					            \end{math}
				      \end{itemize}
			      \end{multicols}



			\item

			      \begin{multicols}{2}
				      \begin{itemize}

					      \item

					            \begin{math}
						            \operatorname{EA}\left(\alert{x}\right)=
						            \left
						            |\alert{x}-
						            \operatorname{fl}\left(\alert{x}\right)
						            \right|=
					            \end{math}

					            \

					      \item

					            \begin{math}
						            \operatorname{ER}\left(\alert{x}\right)=
						            \dfrac{
							            \left|
							            \alert{x}-
							            \operatorname{fl}\left(\alert{x}\right)
							            \right|}{
							            \left|
							            \alert{x}
							            \right|}=
					            \end{math}

					            \

					      \item

					            \begin{math}
						            \operatorname{EA}\left(\alert{y}\right)=
						            \left|
						            \alert{y}-
						            \operatorname{fl}\left(\alert{y}\right)
						            \right|=
					            \end{math}

					            \

					      \item

					            \begin{math}
						            \operatorname{ER}\left(\alert{y}\right)=
						            \dfrac{
							            \left|
							            \alert{y}-
							            \operatorname{fl}\left(\alert{y}\right)
							            \right|}{
							            \left|
							            \alert{y}
							            \right|}=
					            \end{math}
				      \end{itemize}
			      \end{multicols}
		\end{enumerate}
	\end{solution}
\end{frame}

\begin{frame}
	\begin{solution}
		\begin{enumerate}[c)]

			\item

			      \begin{multicols}{2}
				      \begin{itemize}
					      \item

					            \begin{math}
						            \alert{x}+\alert{y}=
						            \operatorname{fl}
						            \left(
						            \operatorname{fl}\left(\alert{x}\right)+
						            \operatorname{fl}\left(\alert{y}\right)
						            \right)=
					            \end{math}

					            \

					      \item

					            \begin{math}
						            \alert{x}-\alert{y}=
						            \operatorname{fl}
						            \left(
						            \operatorname{fl}\left(\alert{x}\right)-
						            \operatorname{fl}\left(\alert{y}\right)
						            \right)=
					            \end{math}

					            \

					      \item

					            \begin{math}
						            \alert{x}\times\alert{y}=
						            \operatorname{fl}
						            \left(
						            \operatorname{fl}\left(\alert{x}\right)\times
						            \operatorname{fl}\left(\alert{y}\right)
						            \right)=
					            \end{math}

					            \

					      \item

					            \begin{math}
						            \dfrac{\alert{x}}{\alert{y}}=
						            \operatorname{fl}
						            \left(
						            \dfrac{
							            \operatorname{fl}\left(\alert{x}\right)
						            }{
							            \operatorname{fl}\left(\alert{y}\right)
						            }
						            \right)=
					            \end{math}
				      \end{itemize}
			      \end{multicols}
		\end{enumerate}

		\begin{enumerate}[d)]
			\item

			      \begin{itemize}
				      \item

				            \begin{math}
					            \operatorname{ER}\left(\alert{x+y}\right)=
					            \dfrac{
						            \left|
						            \left(\alert{x+y}\right)-
						            \operatorname{fl}\left(\alert{x+y}\right)
						            \right|
					            }{
						            \left|
						            \alert{x+y}
						            \right|
					            }=
				            \end{math}

				            \

				            \

				      \item

				            \begin{math}
					            \operatorname{ER}\left(\alert{x-y}\right)=
					            \dfrac{
						            \left|
						            \left(\alert{x-y}\right)-
						            \operatorname{fl}\left(\alert{x-y}\right)
						            \right|
					            }{
						            \left|
						            \alert{x-y}
						            \right|
					            }=
				            \end{math}

				            \

				            \

				      \item

				            \begin{math}
					            \operatorname{ER}
					            \left(\alert{x\times y}\right)=
					            \dfrac{
						            \left|
						            \left(\alert{x\times y}\right)-
						            \operatorname{fl}
						            \left(\alert{x\times y}\right)
						            \right|
					            }{
						            \left|
						            \alert{x\times y}
						            \right|
					            }=
				            \end{math}

				            \

				            \

				      \item

				            \begin{math}
					            \operatorname{ER}
					            \left(\alert{\dfrac{x}{y}}\right)=
					            \dfrac{
						            \left|
						            \left(\alert{\dfrac{x}{y}}\right)-
						            \operatorname{fl}
						            \left(\alert{\dfrac{x}{y}}\right)
						            \right|
					            }{
						            \left|
						            \alert{\dfrac{x}{y}}
						            \right|
					            }=
				            \end{math}
			      \end{itemize}
		\end{enumerate}
	\end{solution}
\end{frame}

\begin{frame}

	\begin{enumerate}\setcounter{enumi}{1}
		\item

		      Si $x=0.43257143$ e $y=0.43257824$.

		      \begin{enumerate}[a)]
			      \item\label{q:2.a}

			      Use aritmética de redondeo a cinco cifras para calcular
			      \begin{math}
				      \operatorname{fl}\left(x\right)
			      \end{math}
			      y
			      \begin{math}
				      \operatorname{fl}\left(y\right)
			      \end{math}.

			      \item\label{q:2.b}

			      Calcular los errores relativos y absolutos.

			      \item\label{q:2.c}

			      Calcular $x+y$, $x-y$, $xy$ y $\frac{x}{y}$.

			      \item\label{q:2.d}

			      Calcular los errores relativos para las operaciones
			      realizadas en el inciso anterior.
		      \end{enumerate}
	\end{enumerate}

	\begin{solution}
		Aldo escribe aquí.
	\end{solution}
\end{frame}

\begin{frame}
	\begin{enumerate}\setcounter{enumi}{4}
		\item

		      Determine el valor decimal, la suma y la diferencia de los
		      números binarios $A=11100111$ y $B=10111111$, suponiendo
		      que:

		      \begin{multicols}{2}

			      \begin{enumerate}[a)]
				      \item

				            Ambos están representados en magnitud y signo.

				      \item

				            Ambos están representados en complemento a 2.
			      \end{enumerate}
		      \end{multicols}
	\end{enumerate}

	\begin{solution}
		.
	\end{solution}
\end{frame}

\begin{frame}
	\begin{enumerate}\setcounter{enumi}{10}
		\item

		      Un computador que usa redondeo y punto flotante con $10$
		      bits posee la siguiente estructura:
		      el primer bit guarda información sobre el signo, los 3 bits
		      siguientes guardan información sobre el exponente
		      (desplazado $3$ unidades) y los $6$ bits restantes guardan
		      los dígitos de la mantisa (a partir del segundo porque el
		      primero siempre es uno y con redondeo en el séptimo dígito
		      si esto es necesario).
		      Por ejemplo, el registro $1110001000$ representa al número
		      ${\left(-1\right)}^{1}\times 0.1001000\times 2^{6-3}$.
		      ¿Cómo almacena este computador al número $9.123$?
		      Calcule el error relativo que se comete al realizar tal
		      representación?
	\end{enumerate}

	\begin{solution}

        
		.
	\end{solution}
\end{frame}

\begin{frame}
	\begin{enumerate}\setcounter{enumi}{13}
		\item

		      Asuma que se necesita calcular
		      \begin{math}
			      A=
			      \sqrt{10^{14}+\frac{2}{3}}-10^{7}
		      \end{math}
		      en un computador que usa aritmética de punto flotante con
		      una exactitud de 15 dígitos.

		      \begin{enumerate}[a)]
			      \item\label{q:14.a}

			      Explicar si esta fórmula producirá información sin
			      pérdida de dígitos significativos.
			      ¿Cuál es el valor?

			      \item\label{q:14.b}

			      Reescribir la fórmula en una forma alternativa de
			      modo que un cálculo más exacto sea posible.
			      Compare con lo obtenido en la parte~\eqref{q:14.a}.
		      \end{enumerate}
	\end{enumerate}

	\begin{solution}
		.
	\end{solution}
\end{frame}

\begin{frame}
	\begin{enumerate}\setcounter{enumi}{15}
		\item

		      Escribir en base dos los siguientes números dados en base $10$.


		      \begin{enumerate}[a)]
			      \item

			            $2324.6$.
		      \end{enumerate}
	\end{enumerate}

	\begin{solution}
		.
	\end{solution}
\end{frame}

\begin{frame}
	\begin{enumerate}\setcounter{enumi}{16}
		\item

		      Si tenemos $\beta=10, N=11$ y $k=6$.
		      Entonces, disponemos de $k=6$ dígitos para la parte
		      fraccionaria y $N-k-1$ dígitos para la parte entera.
		      Escribe la representación de los siguientes números:

		      \begin{enumerate}[b)]
			      \item

			            $40.9561$.
		      \end{enumerate}
	\end{enumerate}

	\begin{solution}
		.
	\end{solution}
\end{frame}

\begin{frame}
	\begin{enumerate}\setcounter{enumi}{18}
		\item

		      Realice las operaciones aritmética con enteros complemento
		      a dos con una longitud de palabra de $N=4$ bits para las
		      siguientes operaciones:

		      \begin{enumerate}[b)]
			      \item

			            $1100+0100$.
		      \end{enumerate}
	\end{enumerate}

	\begin{solution}
		.
	\end{solution}
\end{frame}

\begin{frame}
	\begin{enumerate}\setcounter{enumi}{20}
		\item

		      Sean $a=0.000063381158, b=73.688329$ y $c=-73.687711$.
		      Determine la aritmética de punto flotante para:

		      \begin{enumerate}[b)]
			      \item

			            $\left(a+b\right)+c$.
		      \end{enumerate}
	\end{enumerate}

	\begin{solution}
		.
	\end{solution}
\end{frame}

\section{Pregunta N$^{\circ}$24a\qquad Breiner Catalino Morales}

\begin{frame}
	\begin{enumerate}\setcounter{enumi}{23}
		\item

		      Si tenemos $\beta=2$, $t=3$, $L=-2$ y $U=2$, determine los
		      números de máquina que contiene dicho intervalo y además
		      determine

		      \begin{enumerate}[b)]
			      \item

			            \begin{math}
				            \dfrac{24}{32}\oplus
				            \dfrac{7}{32}
			            \end{math}.
		      \end{enumerate}
	\end{enumerate}

	\begin{solution}
		\begin{definition}[Conjunto de números de punto flotante~\cite{Kincaid1994}]
			Sea el \alert{conjunto de números de punto flotante}
			\begin{equation*}
				\mathbb{F}\left(\beta,t,L,U\right)=
				\left\{0\right\}+
				\left\{
				x\in\mathbb{R}\mid
				x=
				\left(-1\right)^{s}
				\beta^{e}
				\sum_{i=1}^{t}a_{i}\beta^{-i}
				\right\}
			\end{equation*}
			con $t$ dígitos significativos, base $\beta\geq2$,
			$0\leq a_{i}\leq\beta-1$ y rango $\left(L,U\right)$ con
			$0<L\leq e\leq U>0$.
		\end{definition}
	\end{solution}
\end{frame}

\begin{frame}
	\begin{solution}
		Si $x\in\mathbb{F}\left(\beta,t,L,U\right)$, entonces
		% $-x\in\mathbb{F}\left(\beta,t,L,U\right)$ y
		\begin{math}
			x_{\min}=
			\beta^{L-1}\leq
			\left|x\right|\leq
			\beta^{U}\left(1-\beta^{-t}\right)=
			x_{\max}
		\end{math}.
		Para
		\begin{math}
			\mathbb{F}
			\left(
			\beta=2,
			t=3,
			L=-2,
			U=2
			\right)
		\end{math}:
		\begin{equation*}
			\alert{\dfrac{1}{8}}=
			2^{-2-1}=
			x_{\min}\leq
			\left|x\right|\leq
			x_{\max}=
			2^{2}\left(1-2^{-3}\right)=
			4\left(1-\dfrac{1}{8}\right)=
			\alert{\dfrac{7}{2}}.
		\end{equation*}
		\alert{El total de número de máquinas} están dados por los que aparecen en la siguiente tabla,
		sus opuestos y el cero, en total el
		\begin{math}
			\operatorname{card}
			\mathbb{F}
			\left(
			\beta=2,
			t=3,
			L=-2,
			U=2
			\right)=
			2\left(\beta-1\right)
			{\beta}^{t-1}
			\left(U-L+1\right)+1=
			2\left(2-1\right)
			{2}^{3-1}
			\left(
			2-\left(-2\right)+1
			\right)+
			1=
			\alert{41}
		\end{math}.
		% Los números reales $x$ se encuentran en un determinado intervalo
		\begin{table}[ht!]
			\renewcommand{\arraystretch}{2.5}
			\begin{tabular}{|>{$}c<{$}|>{$}c<{$}|>{$}c<{$}|>{$}c<{$}|>{$}c<{$}|>{$}c<{$}|}
				\hline
				-2
				 & -1
				 & 0
				 & 1
				 & 2
				\\
				\hline
				{\left(0.100\right)}_{2}\times 2^{-2}=\alert{\dfrac{1}{8}}
				 & {\left(0.100\right)}_{2}\times 2^{-1}=\dfrac{1}{4}
				 & {\left(0.100\right)}_{2}\times 2^{0}=\dfrac{1}{2}
				 & {\left(0.100\right)}_{2}\times 2^{1}=\dfrac{1}{1}
				 & {\left(0.100\right)}_{2}\times 2^{2}=\dfrac{2}{1}
				\\
				\hline
				{\left(0.101\right)}_{2}\times 2^{-2}=\dfrac{5}{32}
				 & {\left(0.101\right)}_{2}\times 2^{-1}=\dfrac{5}{16}
				 & {\left(0.101\right)}_{2}\times 2^{0}=\dfrac{5}{8}
				 & {\left(0.101\right)}_{2}\times 2^{1}=\dfrac{5}{4}
				 & {\left(0.101\right)}_{2}\times 2^{2}=\dfrac{5}{2}
				\\
				\hline
				{\left(0.110\right)}_{2}\times 2^{-2}=\dfrac{3}{16}
				 & {\left(0.110\right)}_{2}\times 2^{-1}=\dfrac{3}{8}
				 & {\left(0.110\right)}_{2}\times 2^{0}=\dfrac{3}{4}
				 & {\left(0.110\right)}_{2}\times 2^{1}=\dfrac{3}{2}
				 & {\left(0.110\right)}_{2}\times 2^{2}=\dfrac{3}{1}
				\\
				\hline
				{\left(0.111\right)}_{2}\times 2^{-2}=\dfrac{7}{32}
				 & {\left(0.111\right)}_{2}\times 2^{-1}=\dfrac{7}{16}
				 & {\left(0.111\right)}_{2}\times 2^{0}=\dfrac{7}{8}
				 & {\left(0.111\right)}_{2}\times 2^{1}=\dfrac{7}{4}
				 & {\left(0.111\right)}_{2}\times 2^{2}=\alert{\dfrac{7}{2}}
				\\
				\hline
			\end{tabular}
		\end{table}
	\end{solution}
\end{frame}

\begin{frame}
	\begin{align*}
		\dfrac{24}{32}\oplus
		\dfrac{7}{32}
		 & =
		\operatorname{fl}
		\left(
		\operatorname{fl}
		\left(
			\dfrac{24}{32}
			\right)+
		\operatorname{fl}
		\left(
			\dfrac{7}{32}
			\right)
		\right) \\
		 & =
		\operatorname{fl}
		\left(
		\dfrac{24}{32}
		+
		\dfrac{7}{32}
		\right) \\
		 & =
		\operatorname{fl}
		\left(
		\dfrac{31}{32}
		\right) \\
		 & =1.
	\end{align*}
\end{frame}
\input{references}

\end{document}