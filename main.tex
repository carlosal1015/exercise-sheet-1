%! Aldo Luna Bueno
%! Alejandro Maycoll Escobar Mejia
%! Carlos Alonso Aznarán Laos
%! Khalid Zaid Izquierdo Ayllon
%! Brian Alberto Huaman Garcia
%! Carlos Daniel Malvaceda Canales
%! Bruno Cipriano Arroyo
%! Breiner Catalino Morales
%! Universidad Nacional de Ingeniería
%! Facultad de Ciencias
%! Lima, Perú
%! Uso:
%! $ sudo pacman -Syu texlive-most zathura # dependencias, visor
%! $ arara solution_primera
%! $ zathura solution_primera.pdf
%! Ver https://wiki.archlinux.org/title/TeX_Live
% arara: clean: {
% arara: --> extensions:
% arara: --> ['aux','log','nav',
% arara: --> 'out','snm','toc','pytxcode','pdf']
% arara: --> }
% arara: lualatex: {
% arara: --> shell: yes,
% arara: --> draft: yes,
% arara: --> interaction: batchmode
% arara: --> }
% arara: biber
% arara: lualatex: {
% arara: --> shell: yes,
% arara: --> draft: yes,
% arara: --> interaction: batchmode
% arara: --> }
% arara: lualatex: {
% arara: --> shell: yes,
% arara: --> synctex: yes,
% arara: --> interaction: batchmode
% arara: --> }
% arara: clean: {
% arara: --> extensions:
% arara: --> ['aux','log','nav',
% arara: --> 'out','snm','toc','pytxcode']
% arara: --> }
\PassOptionsToPackage{svgnames}{xcolor}
\documentclass[
	spanish,
	8pt,
	utf8,
	xcolor=table,
	handout,
	aspectratio=169,
	professionalfonts,
	% notheorems,
	mathserif,
	leqno,
	% t
]{beamer}
\setbeamersize{text margin left=5pt,text margin right=5pt}
\usepackage[spanish,es-sloppy]{babel}
\spanishdatedel\decimalpoint
\usepackage{mathtools}
\usepackage{minted}
\usepackage{enumerate}
\usepackage{multicol}
% \usepackage{pythontex}
\usepackage[
	citestyle=numeric,
	style=apa,
	backend=biber,
	defernumbers=true,
	sorting=ynt,
	maxcitenames=4
]{biblatex}
\addbibresource{references.bib}

\newcolumntype{x}[1]{>{\centering\arraybackslash\hspace{0pt}}p{#1}}

\newcounter{savedenum}
\newcommand*{\saveenum}{\setcounter{savedenum}{\theenumi}}
\newcommand*{\resume}{\setcounter{enumi}{\thesavedenum}}

\setbeamertemplate{navigation symbols}{}
\setbeamertemplate{footline}{}
\setbeamertemplate{headline}{}

% https://tex.stackexchange.com/questions/68080/beamer-bibliography-icon
\setbeamertemplate{bibliography item}{%
	\ifboolexpr{ test {\ifentrytype{book}} or test {\ifentrytype{mvbook}}
		or test {\ifentrytype{collection}} or test {\ifentrytype{mvcollection}}
		or test {\ifentrytype{reference}} or test {\ifentrytype{mvreference}} }
	{\setbeamertemplate{bibliography item}[book]}
	{\ifentrytype{online}
		{\setbeamertemplate{bibliography item}[online]}
		{\setbeamertemplate{bibliography item}[article]}}%
	\usebeamertemplate{bibliography item}}

\defbibenvironment{bibliography}
{\list{}
	{\settowidth{\labelwidth}{\usebeamertemplate{bibliography item}}%
		\setlength{\leftmargin}{\labelwidth}%
		\setlength{\labelsep}{\biblabelsep}%
		\addtolength{\leftmargin}{\labelsep}%
		\setlength{\itemsep}{\bibitemsep}%
		\setlength{\parsep}{\bibparsep}}}
{\endlist}
{\item}

\title{
	\huge\sffamily
	Primera Práctica Dirigida\quad Grupo N$^{\circ}$~1
}

\subtitle{
	\large\scshape
	Análisis y Modelamiento Numérico I\quad CM4F1 B\\[.5\baselineskip]
		\normalsize\normalfont
		Profesor Ángel Enrique Ramírez Gutiérrez.
}

\author{
	Aldo Luna Bueno\quad\and\quad
	Alejandro Escobar Mejia\quad\and\quad
	Carlos Aznarán Laos\quad\and\quad
	Brian Huaman Garcia\and \\[\baselineskip]
  Khalid Izquierdo Ayllon\quad\and\quad
  Carlos Malvaceda Canales\quad\and\quad
  Bruno Cipriano Arroyo\quad\and\quad
  Breiner Catalino Morales
}

\institute{\large
	Facultad de Ciencias \and
	Universidad Nacional de Ingeniería
}
\date{5 de abril del 2023}

\begin{document}

\frame{\titlepage}

\begin{frame}

	\begin{enumerate}\setcounter{enumi}{1}
		\item

		      Si $x=0.43257143$ e $y=0.43257824$.

		      \begin{enumerate}[a)]
			      \item\label{q:2.a}

			      Use aritmética de redondeo a cinco cifras para calcular
			      \begin{math}
				      \operatorname{fl}\left(x\right)
			      \end{math}
			      y
			      \begin{math}
				      \operatorname{fl}\left(y\right)
			      \end{math}.

			      \item\label{q:2.b}

			      Calcular los errores relativos y absolutos.

			      \item\label{q:2.c}

			      Calcular $x+y$, $x-y$, $xy$ y $\frac{x}{y}$.

			      \item\label{q:2.d}

			      Calcular los errores relativos para las operaciones
			      realizadas en el inciso anterior.
		      \end{enumerate}
	\end{enumerate}

	\begin{solution}
		.
	\end{solution}
\end{frame}

\begin{frame}
	\begin{enumerate}\setcounter{enumi}{4}
		\item

		      Determine el valor decimal, la suma y la diferencia de los
		      números binarios $A=11100111$ y $B=10111111$, suponiendo
		      que:

		      \begin{multicols}{2}

			      \begin{enumerate}[a)]
				      \item

				            Ambos están representados en magnitud y signo.

				      \item

				            Ambos están representados en complemento a 2.
			      \end{enumerate}
		      \end{multicols}
	\end{enumerate}

	\begin{solution}
		.
	\end{solution}
\end{frame}

\begin{frame}
	\begin{enumerate}\setcounter{enumi}{10}
		\item

		      Un computador que usa redondeo y punto flotante con $10$
		      bits posee la siguiente estructura:
		      el primer bit guarda información sobre el signo, los 3 bits
		      siguientes guardan información sobre el exponente
		      (desplazado $3$ unidades) y los $6$ bits restantes guardan
		      los dígitos de la mantisa (a partir del segundo porque el
		      primero siempre es uno y con redondeo en el séptimo dígito
		      si esto es necesario).
		      Por ejemplo, el registro $1110001000$ representa al número
		      ${\left(-1\right)}^{1}\times 0.1001000\times 2^{6-3}$.
		      ¿Cómo almacena este computador al número $9.123$?
		      Calcule el error relativo que se comete al realizar tal
		      representación?
	\end{enumerate}

	\begin{solution}
		.
	\end{solution}
\end{frame}

\begin{frame}
	\begin{enumerate}\setcounter{enumi}{13}
		\item

		      Asuma que se necesita calcular
		      \begin{math}
			      A=
			      \sqrt{10^{14}+\frac{2}{3}}-10^{7}
		      \end{math}
		      en un computador que usa aritmética de punto flotante con
		      una exactitud de 15 dígitos.

		      \begin{enumerate}[a)]
			      \item\label{q:14.a}

			      Explicar si esta fórmula producirá información sin
			      pérdida de dígitos significativos.
			      ¿Cuál es el valor?

			      \item\label{q:14.b}

			      Reescribir la fórmula en una forma alternativa de
			      modo que un cálculo más exacto sea posible.
			      Compare con lo obtenido en la parte~\eqref{q:14.a}.
		      \end{enumerate}
	\end{enumerate}

	\begin{solution}
		.
	\end{solution}
\end{frame}

\begin{frame}
	\begin{enumerate}\setcounter{enumi}{15}
		\item

		      Escribir en base dos los siguientes números dados en base $10$.


		      \begin{enumerate}[a)]
			      \item

			            $2324.6$.
		      \end{enumerate}
	\end{enumerate}

	\begin{solution}
		.
	\end{solution}
\end{frame}

\begin{frame}
	\begin{enumerate}\setcounter{enumi}{16}
		\item

		      Si tenemos $\beta=10, N=11$ y $k=6$.
		      Entonces, disponemos de $k=6$ dígitos para la parte
		      fraccionaria y $N-k-1$ dígitos para la parte entera.
		      Escribe la representación de los siguientes números:

		      \begin{enumerate}[b)]
			      \item

			            $40.9561$.
		      \end{enumerate}
	\end{enumerate}

	\begin{solution}
		.
	\end{solution}
\end{frame}

\begin{frame}
	\begin{enumerate}\setcounter{enumi}{18}
		\item

		      Realice las operaciones aritmética con enteros complemento
		      a dos con una longitud de palabra de $N=4$ bits para las
		      siguientes operaciones:

		      \begin{enumerate}[b)]
			      \item

			            $1100+0100$.
		      \end{enumerate}
	\end{enumerate}

	\begin{solution}
		.
	\end{solution}
\end{frame}

\begin{frame}
	\begin{enumerate}\setcounter{enumi}{20}
		\item

		      Sean $a=0.000063381158, b=73.688329$ y $c=-73.687711$.
		      Determine la aritmética de punto flotante para:

		      \begin{enumerate}[b)]
			      \item

			            $\left(a+b\right)+c$.
		      \end{enumerate}
	\end{enumerate}

	\begin{solution}
		.
	\end{solution}
\end{frame}

\begin{frame}
	\begin{enumerate}\setcounter{enumi}{23}
		\item

		      Si tenemos $\beta=2$, $t=3$, $L=-2$ y $U=2$, determine los
		      números de máquina que contiene dicho intervalo y además
		      determine:

		      \begin{enumerate}[b)]
			      \item

			            $\frac{24}{32}\oplus\frac{7}{32}$.
		      \end{enumerate}
	\end{enumerate}

	\begin{solution}
		.
	\end{solution}
\end{frame}

\begin{frame}\transblindsvertical
	\frametitle{Referencias}
	%------------------------------------------------------------ 1
	\only<1>{
		\begin{itemize}
			\item Libros
			      \nocite{*}
			      \printbibliography[heading=none,keyword=book]
		\end{itemize}
	}
	%------------------------------------------------------------ 2
	\only<2>{
		\begin{itemize}
			\item Artículos científicos
			      \printbibliography[heading=none,keyword=paper]
		\end{itemize}
	}
	%------------------------------------------------------------ 3
	\only<3>{
		\begin{itemize}
			\item Sitios web
			      \printbibliography[heading=none,keyword=online]
		\end{itemize}
	}
\end{frame}

\end{document}