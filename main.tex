%! Aldo Luna Bueno
%! Alejandro Maycoll Escobar Mejia
%! Carlos Alonso Aznarán Laos
%! Luis Enrique Hernandez Pino
%! Khalid Zaid Izquierdo Ayllon
%! Carlos Daniel Malvaceda Canales
%! Bruno Cipriano Arroyo
%! Breiner Catalino Morales
%! Universidad Nacional de Ingeniería
%! Facultad de Ciencias
%! Lima, Perú
%! Uso:
%! $ sudo pacman -Syu texlive-most zathura # dependencias, visor
%! $ arara solution_primera
%! $ zathura solution_primera.pdf
%! Ver https://wiki.archlinux.org/title/TeX_Live
% arara: clean: {
% arara: --> extensions:
% arara: --> ['aux','log','nav',
% arara: --> 'out','snm','toc','pytxcode','pdf']
% arara: --> }
% arara: lualatex: {
% arara: --> shell: yes,
% arara: --> draft: yes,
% arara: --> interaction: batchmode
% arara: --> }
% arara: biber
% arara: lualatex: {
% arara: --> shell: yes,
% arara: --> draft: yes,
% arara: --> interaction: batchmode
% arara: --> }
% arara: lualatex: {
% arara: --> shell: yes,
% arara: --> synctex: yes,
% arara: --> interaction: batchmode
% arara: --> }
% arara: clean: {
% arara: --> extensions:
% arara: --> ['aux','log','nav',
% arara: --> 'out','snm','toc','pytxcode']
% arara: --> }
\PassOptionsToPackage{svgnames}{xcolor}
\documentclass[
	spanish,
	8pt,
	utf8,
	xcolor=table,
	handout,
	aspectratio=169,
	professionalfonts,
	% notheorems,
	mathserif,
	leqno,
	% t
]{beamer}
\setbeamersize{text margin left=5pt,text margin right=5pt}
\usepackage[spanish,es-sloppy]{babel}
\spanishdatedel\decimalpoint
\usepackage{mathtools}
\usepackage{minted}
\usepackage{enumerate}
\usepackage{multicol}
% \usepackage{pythontex}
\usepackage[
	citestyle=numeric,
	style=apa,
	backend=biber,
	defernumbers=true,
	sorting=ynt,
	maxcitenames=4
]{biblatex}
\addbibresource{references.bib}

\newcolumntype{x}[1]{>{\centering\arraybackslash\hspace{0pt}}p{#1}}

\newcounter{savedenum}
\newcommand*{\saveenum}{\setcounter{savedenum}{\theenumi}}
\newcommand*{\resume}{\setcounter{enumi}{\thesavedenum}}

\setbeamertemplate{navigation symbols}{}
\setbeamertemplate{footline}{}
\setbeamertemplate{headline}{}

% https://tex.stackexchange.com/questions/68080/beamer-bibliography-icon
\setbeamertemplate{bibliography item}{%
	\ifboolexpr{ test {\ifentrytype{book}} or test {\ifentrytype{mvbook}}
		or test {\ifentrytype{collection}} or test {\ifentrytype{mvcollection}}
		or test {\ifentrytype{reference}} or test {\ifentrytype{mvreference}} }
	{\setbeamertemplate{bibliography item}[book]}
	{\ifentrytype{online}
		{\setbeamertemplate{bibliography item}[online]}
		{\setbeamertemplate{bibliography item}[article]}}%
	\usebeamertemplate{bibliography item}}

\defbibenvironment{bibliography}
{\list{}
	{\settowidth{\labelwidth}{\usebeamertemplate{bibliography item}}%
		\setlength{\leftmargin}{\labelwidth}%
		\setlength{\labelsep}{\biblabelsep}%
		\addtolength{\leftmargin}{\labelsep}%
		\setlength{\itemsep}{\bibitemsep}%
		\setlength{\parsep}{\bibparsep}}}
{\endlist}
{\item}

\title{
	\huge\sffamily
	Primera Práctica Dirigida\quad Grupo N$^{\circ}$~1
}

\subtitle{
	\large\scshape
	Análisis y Modelamiento Numérico I\quad CM4F1 B\\[.5\baselineskip]
		\normalsize\normalfont
		Profesor Ángel Enrique Ramírez Gutiérrez.
}

\author{
	Aldo Luna Bueno\quad\and\quad
	Alejandro Escobar Mejia\quad\and\quad
	Carlos Aznarán Laos\quad\and\quad
	Luis Hernandez Pino\and \\[\baselineskip]
  Khalid Izquierdo Ayllon\quad\and\quad
  Carlos Malvaceda Canales\quad\and\quad
  Bruno Cipriano Arroyo\quad\and\quad
  Breiner Catalino Morales
}

\institute{\large
	Facultad de Ciencias \and
	Universidad Nacional de Ingeniería
}
\date{5 de abril del 2023}

\begin{document}

\frame{\titlepage}

\begin{frame}

	\begin{enumerate}\setcounter{enumi}{1}
		\item

		      Si $x=0.43257143$ e $y=0.43257824$.

		      \begin{enumerate}[a)]
			      \item\label{q:2.a}

			      Use aritmética de redondeo a cinco cifras para calcular
			      \begin{math}
				      \operatorname{fl}\left(x\right)
			      \end{math}
			      y
			      \begin{math}
				      \operatorname{fl}\left(y\right)
			      \end{math}.

			      \item\label{q:2.b}

			      Calcular los errores relativos y absolutos.

			      \item\label{q:2.c}

			      Calcular $x+y$, $x-y$, $xy$ y $\frac{x}{y}$.

			      \item\label{q:2.d}

			      Calcular los errores relativos para las operaciones
			      realizadas en el inciso anterior.
		      \end{enumerate}
	\end{enumerate}

	\begin{solution}
		.
	\end{solution}
\end{frame}

\begin{frame}
	\begin{enumerate}\setcounter{enumi}{4}
		\item

		      Determine el valor decimal, la suma y la diferencia de los
		      números binarios $A=11100111$ y $B=10111111$, suponiendo
		      que:

		      \begin{multicols}{2}

			      \begin{enumerate}[a)]
				      \item

				            Ambos están representados en magnitud y signo.

				      \item

				            Ambos están representados en complemento a 2.
			      \end{enumerate}
		      \end{multicols}
	\end{enumerate}

	\begin{solution}
		.
	\end{solution}
\end{frame}

% \begin{frame}
% 	\begin{enumerate}\setcounter{enumi}{19}
% 		\item

% 		      Probar que el cardinal del sistema de números de punto
% 		      flotante normalizado
% 		      $\mathbb{F}\left(\beta, t, L, U\right)$ es

% 		      \begin{equation*}
% 			      \operatorname{card}\left(\mathbb{F}\right)=
% 			      2\left(\beta-1\right)\beta^{t-1}\left(U-L+1\right)+1.
% 		      \end{equation*}

% 		      En particular, para $\mathbb{F}\left(10,3,-2,3\right)$,
% 		      calcular $x_{\min}$, $x_{\max}$, $\epsilon_{M}$.

% 	\end{enumerate}
% \end{frame}

% \begin{frame}
% 	\begin{definition}[Representación de
% 			punto flotante]
% 		Un número real no nulo $x$, tiene como
% 		\alert{representación de punto flotante} a
% 		\begin{equation*}
% 			x=
% 			{\left(-1\right)}^{s}\cdot
% 			\left(0.a_{1}\ldots a_{t}\right)\cdot
% 			{\beta}^{e}=
% 			{\left(-1\right)}^{s}\cdot
% 			m\cdot{\beta}^{e-t},
% 		\end{equation*}
% 		donde $t\in\mathbb{N}$ es el número de dígitos significativos
% 		permitidos $a_{i}$ (con $0\leq a_{i}\leq\beta-1$),
% 		$m=a_{1}\ldots a_{t}$ es un número entero llamado mantisa tal
% 		que $0\leq m\leq{\beta}^{t}-1$ y $e$ es número entero
% 		llamado exponente, donde $0>L\leq e\leq U>0$.
% 	\end{definition}

% 	\begin{definition}[El conjunto de los números punto flotante]
% 		El conjunto de los números punto flotante con $t$ dígitos
% 		significativos, base $\beta\geq2$, $0\leq a_{i}\leq\beta-1$ y
% 		$L\leq e\leq U$.
% 		\begin{equation*}
% 			\mathbb{F}\left(\beta, t, L, U\right)=
% 			\left\{0\right\}\cup
% 			\left\{
% 			x\in\mathbb{R}:x=
% 			{\left(-1\right)}^{s}
% 			\beta^{e}
% 			\sum_{i=1}^{t}a_{i}\beta^{-i}
% 			\right\}.
% 		\end{equation*}
% 	\end{definition}
% 	Con el fin de tener una representación numérica única, asumimos
% 	que $a_{i}\neq0$ y $m\geq{\beta}^{t-1}$.
% 	$a_{1}$ es el dígito significativo principal, $a_{t}$ es el
% 	último dígito significativo y la
% 	representación de $x$ es llamado \alert{normalizado}.
% \end{frame}

% \begin{frame}
% 	\begin{solution}
% 		Sea el sistema de punto flotante normalizado, un elemento de
% 		$\mathbb{F}\left(\beta, t, L, U\right)$ es de la forma
% 		\begin{equation*}
% 			x=
% 			{\left(-1\right)}^{s}\cdot
% 			\left(0.a_{1}\ldots a_{t}\right)\cdot
% 			{\beta}^{e}.
% 		\end{equation*}

% 		Notamos que

% 		\begin{itemize}
% 			\item

% 			      El signo $s\in\left\{0,1\right\}$ toma dos valores.

% 			\item

% 			      Los dígitos significativos
% 			      $a_{i}\in\left\{0,\ldots,\beta-1\right\}$ con
% 			      $a_{1}\neq 0$ e $i\in\left\{1,\ldots,t\right\}$.

% 			\item

% 			      El exponente $e\in\left\{L,\ldots,U\right\}$ toma $U-L+1$ valores.
% 		\end{itemize}

% 		Por el principio de la multiplicación, la cantidad total de
% 		términos son
% 		\begin{equation*}
% 			\operatorname{card}\left(\mathbb{F}\right)=
% 			2\times
% 			\left(\beta-1\right)\times
% 			\beta^{t-1}\times
% 			\left(U-L+1\right)+
% 			\operatorname{card}\left(\left\{0\right\}\right)=
% 			2\left(\beta-1\right)\beta^{t-1}\left(U-L+1\right)+
% 			1.
% 		\end{equation*}

% 		Por otro lado, las cotas inferior y superior del valor absoluto
% 		de $x$ son
% 		\begin{align*}
% 			x_{\min} & =
% 			{\left(-1\right)}^{0}\cdot
% 			\left(0.1\ldots 0\right)\cdot
% 			{\beta}^{L}=
% 			1\cdot
% 			{\beta}^{-1}\cdot
% 			{\beta}^{L}=
% 			{\beta}^{-1+L}=
% 			\beta^{L-1}\leq
% 			\left|x\right|. \\
% 			x_{\max} & =
% 			{\left(-1\right)}^{0}\cdot
% 			\left(0.\beta-1\ldots \beta-1\right)\cdot
% 			{\beta}^{U}=
% 			1\cdot
% 			\sum_{i=1}^{t}
% 			\left(\beta-1\right)
% 			\beta^{-i}\cdot
% 			{\beta}^{U}=
% 			\left(\beta-1\right)
% 			{\beta}^{U}
% 			\sum_{i=1}^{t}
% 			\beta^{-i}.     \\
% 			x_{\max} & =
% 			\left(\beta-1\right)
% 			{\beta}^{U}
% 			\left(\frac{1-\beta^{-t}}{\beta-1}\right)
% 			=\beta^{U}\left(1-\beta^{-t}\right)\geq
% 			\left|x\right|.
% 		\end{align*}
% 	\end{solution}
% \end{frame}

% \begin{frame}
% 	\begin{solution}
% 		Consideremos el conjunto de punto flotante normalizado
% 		$\mathbb{F}\left(10,3,-2,3\right)$, aplicando la fórmula deducida
% 		obtenemos que el
% 		\begin{equation*}
% 			\operatorname{card}
% 			\left(
% 			\mathbb{F}
% 			\left(\beta=10,t=3,L=-2,U=3\right)
% 			\right)=2\times
% 			\left(10-1\right)\times
% 			10^{3-1}\times
% 			\left(3-\left(-2\right)+1\right)+
% 			1=
% 			2\times 9\times 10^{2}\times 6 + 1
% 			=10801.
% 		\end{equation*}

% 		También,
% 		\begin{align*}
% 			x_{\min}     & =
% 			\beta^{L-1}=
% 			10^{-2-1}=
% 			10^{-3}.         \\
% 			x_{\max}     & =
% 			\beta^{U}\left(1-\beta^{-t}\right)=
% 			10^{3}\left(1-10^{-3}\right)=
% 			999.             \\
% 			\epsilon_{M} & =
% 			\beta^{1-U}=
% 			10^{1-3}=10^{-2}.
% 		\end{align*}
% 	\end{solution}
% \end{frame}

% \begin{frame}\transblindsvertical
% 	\frametitle{Referencias}
% 	%------------------------------------------------------------ 1
% 	\only<1>{
% 		\begin{itemize}
% 			\item Libros
% 			      \nocite{*}
% 			      \printbibliography[heading=none,keyword=book]
% 		\end{itemize}
% 	}
% 	%------------------------------------------------------------ 2
% 	\only<2>{
% 		\begin{itemize}
% 			\item Artículos científicos
% 			      \printbibliography[heading=none,keyword=paper]
% 		\end{itemize}
% 	}
% 	%------------------------------------------------------------ 3
% 	\only<3>{
% 		\begin{itemize}
% 			\item Sitios web
% 			      \printbibliography[heading=none,keyword=online]
% 		\end{itemize}
% 	}
% \end{frame}

\end{document}