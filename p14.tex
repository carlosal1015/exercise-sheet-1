\section{Pregunta N$^{\circ}$14\qquad Alejandro Escobar Mejia}

\begin{frame}
	\begin{enumerate}\setcounter{enumi}{13}
		\item

		      Asuma que se necesita calcular
		      \begin{math}
			      A=
			      \sqrt{10^{14}+\dfrac{2}{3}}-10^{7}
		      \end{math}
		      en un computador que usa aritmética de punto flotante con
		      una exactitud de 15 dígitos.

		      \begin{enumerate}[a)]
			      \item\label{q:14.a}

			      Explicar si esta fórmula producirá información sin
			      pérdida de dígitos significativos.
			      ¿Cuál es el valor?

			      \item\label{q:14.b}

			      Reescribir la fórmula en una forma alternativa de
			      modo que un cálculo más exacto sea posible.
			      Compare con lo obtenido en la parte~\eqref{q:14.a}.
		      \end{enumerate}
	\end{enumerate}

	\begin{solution}

		\begin{equation*}
			\forall a,b\in\mathbb{R}:
			a^{2}-b^{2}=
			\left(a-b\right)
			\left(a+b\right).
		\end{equation*}

		\begin{enumerate}[a)]
			\item
        
         
            \begin{equation*}
	      		  
      
                     M= 10^{14}=0.01*10^{16}  \\
                     N= \dfrac{2}{3}=0.666666666666667=0.0000000000000000666666666666667*10^{16} \\
                      M+N=0.0100000000000000666666666666667*10^{16}\\
                    \sqrt{M+N}=\sqrt{0.0100000000000000666666666666667*10^{16}}=0.1*10^{8}\\
                    10^{7}=0.1*10^{8}\\
                    A=0.1*10^{8}-0.1*10^{8}=0\\
                    \text{Se genera una perdida de información ya que el valor real es} 0.000000033527613
				      
                    
		  \end{equation*}  
    
			\item
			      \begin{equation*}
				      A=
				      \sqrt{10^{14}+\dfrac{2}{3}}-10^{7}\cdot
				      \alert{
				      \dfrac{
				      \sqrt{10^{14}+\dfrac{2}{3}}+10^{7}
				      }{
				      \sqrt{10^{14}+\dfrac{2}{3}}+10^{7}
				      }
				      }=
				      \dfrac{
				      \left(
				      \sqrt{10^{14}+\dfrac{2}{3}}
				      \right)^{2}
				      -{\left(
				      10^{7}
				      \right)}^{2}
				      }{
				      \sqrt{10^{14}+\dfrac{2}{3}}+10^{7}
				      }
				      =
				      \dfrac{
				      10^{14}+\dfrac{2}{3}-10^{14}
				      }{
				      \sqrt{10^{14}+\dfrac{2}{3}}+10^{7}
				      }=
				      \dfrac{\dfrac{2}{3}}{
				      \sqrt{10^{14}+\dfrac{2}{3}}+10^{7}
				      }.
			      \end{equation*}
		\end{enumerate}
	\end{solution}
\end{frame}
