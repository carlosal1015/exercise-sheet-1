\section{Pregunta N$^{\circ}$14\qquad Brian Huaman Garcia}

\begin{frame}
	\begin{enumerate}\setcounter{enumi}{13}
		\item

		      Asuma que se necesita calcular
		      \begin{math}
			      A=
			      \sqrt{10^{14}+\frac{2}{3}}-10^{7}
		      \end{math}
		      en un computador que usa aritmética de punto flotante con
		      una exactitud de 15 dígitos.

		      \begin{enumerate}[a)]
			      \item\label{q:14.a}

			      Explicar si esta fórmula producirá información sin
			      pérdida de dígitos significativos.
			      ¿Cuál es el valor?

			      \item\label{q:14.b}

			      Reescribir la fórmula en una forma alternativa de
			      modo que un cálculo más exacto sea posible.
			      Compare con lo obtenido en la parte~\eqref{q:14.a}.
		      \end{enumerate}
	\end{enumerate}

	\begin{solution}
		.
	\end{solution}
\end{frame}
