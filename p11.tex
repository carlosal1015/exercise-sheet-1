\section{Pregunta N$^{\circ}$11\qquad Aldo Luna Bueno}

\begin{frame}
	\begin{enumerate}\setcounter{enumi}{10}
		\item

		      Un computador que usa redondeo y punto flotante con $10$
		      bits posee la siguiente estructura:
		      el primer bit guarda información sobre el signo, los 3 bits
		      siguientes guardan información sobre el exponente
		      (desplazado $3$ unidades) y los $6$ bits restantes guardan
		      los dígitos de la mantisa (a partir del segundo porque el
		      primero siempre es uno y con redondeo en el séptimo dígito
		      si esto es necesario).
		      Por ejemplo, el registro $1110001000$ representa al número
		      ${\left(-1\right)}^{1}\times 0.1001000\times 2^{6-3}$.

		      \begin{enumerate}[a)]
			      \item

			            ¿Cómo almacena este computador el número $9.123$?

			      \item

			            Calcule el error relativo que se comete al realizar
			            tal representación.
		      \end{enumerate}
	\end{enumerate}

	\begin{solution}
		\begin{enumerate}[a)]
			\item
			      Sea $N$ = 9.123 el número expresado en sistema decimal.

			      \begin{columns}[t]
				      \begin{column}{0.48\textwidth}
					      \begin{description}
						      \item[Parte entera de $N$]

							      $n_{0}=9$

							      \begin{table}[ht!]
								      \begin{tabular}{>{$}c<{$}|>{$}c<{$}|>{$}c<{$}}
									      k & n_{k} & a_{k} \\
									      \hline
									      0 & 9     & 1     \\
									      1 & 4     & 0     \\
									      2 & 2     & 0     \\
									      3 & 1     & 1
								      \end{tabular}
							      \end{table}
					      \end{description}
				      \end{column}
				      \begin{column}{0.48\textwidth}
					      \begin{description}
						      \item[Parte fraccionaria de $N$]

							      $r_{1}=0.123$

							      \begin{table}[ht!]
								      \begin{tabular}{>{$}c<{$}|>{$}c<{$}|>{$}c<{$}|>{$}c<{$}}
									      k & r_{k} & 2r_{k} & d_{k} \\
									      \hline
									      1 & 0.123 & 0.246  & 0     \\
									      2 & 0.246 & 0.492  & 0     \\
									      3 & 0.492 & 0.984  & 0     \\
									      4 & 0.984 & 1.968  & 1     \\
									      5 & 0.968 & 1.936  & 1
								      \end{tabular}
							      \end{table}

					      \end{description}
				      \end{column}
			      \end{columns}

			      Entonces,
			      \begin{math}
				      N=
				      {1001.00011\dotsc}_{\left(2\right)}
			      \end{math}.

			      Su representación en notación científica normalizada
			      sería
			      \begin{math}
				      N=
				      {0.1(001000)11\dotsc}_{\left(2\right)}\times
				      2^{4}
			      \end{math}.
		\end{enumerate}
	\end{solution}
\end{frame}

\begin{frame}
	\begin{solution}
		\begin{enumerate}[a)]
			\item

			      Haremos el redondeo con ayuda de los paréntesis.
			      Estos separan los lugares de los seis bits que irán a la
			      mantisa.
			      Como el bit que sigue inmediatamente después del séptimo
			      es $1$ y luego hay al menos otro $1$, el séptimo bit
			      aumenta $1$:
			      \begin{math}
				      N=0.1\left(001001\right)_{2}\times
				      2^{4}
			      \end{math}

			      Ahora añadimos a la representación el signo y el desplazamiento
			      del exponente (-3):

			      \begin{math}
				      N=
				      {\left(-1\right)}^{0}\times
				      0.1\left(001001\right)_{\left(2\right)}\times
				      2^{7-3}
			      \end{math}

			      Signo: 0
			      Mantisa: 001001
			      Exponente desplazado: 7 = $111_{2}$

			      Por lo tanto, una computadora con redondeo y punto flotante de
			      $10$ bits, como se ha descrito, almacena el número $9.123$ de
			      esta forma:

			      \begin{table}[ht!]
				      \begin{tabular}{|>{$}c<{$}|>{$}c<{$} >{$}c<{$} >{$}c<{$}|>{$}c<{$} >{$}c<{$} >{$}c<{$} >{$}c<{$} >{$}c<{$} >{$}c<{$}|}
					      \hline
					      0 & 1 & 1 & 1 & 0 & 0 & 1 & 0 & 0 & 1 \\
					      \hline
				      \end{tabular}
			      \end{table}

			\item

			      .
		\end{enumerate}
	\end{solution}
\end{frame}