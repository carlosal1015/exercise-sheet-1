\section{Pregunta N$^{\circ}$11\qquad Carlos Aznarán Laos}

\begin{frame}
	\begin{enumerate}\setcounter{enumi}{10}
		\item

		      Un computador que usa redondeo y punto flotante con $10$
		      bits posee la siguiente estructura:
		      el primer bit guarda información sobre el signo, los 3 bits
		      siguientes guardan información sobre el exponente
		      (desplazado $3$ unidades) y los $6$ bits restantes guardan
		      los dígitos de la mantisa (a partir del segundo porque el
		      primero siempre es uno y con redondeo en el séptimo dígito
		      si esto es necesario).
		      Por ejemplo, el registro $1110001000$ representa al número
		      ${\left(-1\right)}^{1}\times 0.1001000\times 2^{6-3}$.
		      ¿Cómo almacena este computador al número $9.123$?
		      Calcule el error relativo que se comete al realizar tal
		      representación?
	\end{enumerate}

	\begin{solution}

		\begin{table}[ht!]
			\begin{tabular}{|>{$}c<{$}|>{$}c<{$}|>{$}c<{$}|>{$}c<{$}|>{$}c<{$}|>{$}c<{$}|>{$}c<{$}|>{$}c<{$}|>{$}c<{$}|>{$}c<{$}|}
				\hline
				\alert{1} & 1 & 1 & 0 & 0 & 0 & 1 & 0 & 0 & 0 \\
				\hline
			\end{tabular}
		\end{table}

		.
	\end{solution}
\end{frame}