\section{Pregunta N$^{\circ}$16a\qquad Khalid Izquierdo Ayllon}

\begin{frame}
	\begin{enumerate}\setcounter{enumi}{15}
		\item

		      Escribir en base dos los siguientes números dados en base $10$.


		      \begin{enumerate}[a)]
			      \item
                    \begin{align*}
			            \text{Parte entera: } & \\
			            2324_{10} \div 2 &= 1162 \text{ resto } 0 \\
			            1162 \div 2 &= 581 \text{ resto } 0 \\
			            581 \div 2 &= 290 \text{ resto } 1 \\
			            290 \div 2 &= 145 \text{ resto } 0 \\
			            145 \div 2 &= 72 \text{ resto } 1 \\
			            72 \div 2 &= 36 \text{ resto } 0 \\
			            36 \div 2 &= 18 \text{ resto } 0 \\
			            18 \div 2 &= 9 \text{ resto } 0 \\
			            9 \div 2 &= 4 \text{ resto } 1 \\
			            4 \div 2 &= 2 \text{ resto } 0 \\
			            2 \div 2 &= 1 \text{ resto } 0 \\
			            1 \div 2 &= 0 \text{ resto } 1 \\
			            \text{Parte entera en base 2: } & 100100010100 \\
			            \\
			            \text{Parte fraccionaria: } & \\
			            0.6 \times 2 &= 1.2 \quad (\text{parte entera 1}) \\
			            0.2 \times 2 &= 0.4 \quad (\text{parte entera 0}) \\
			            0.4 \times 2 &= 0.8 \quad (\text{parte entera 0}) \\
			            0.8 \times 2 &= 1.6 \quad (\text{parte entera 1}) \\
			            0.6 \times 2 &= 1.2 \quad (\text{parte entera 1}) \\
			            0.2 \times 2 &= 0.4 \quad (\text{parte entera 0}) \\
			            0.4 \times 2 &= 0.8 \quad (\text{parte entera 0}) \\
			            0.8 \times 2 &= 1.6 \quad (\text{parte entera 1}) \\
			            \text{Parte fraccionaria en base 2: } & 0.1001\overline{1001} \\
			            \\
			            \text{Número completo en base 2: } & 100100010100.1001\overline{1001}
		            \end{align*}

	      \end{enumerate}
\end{enumerate}

\begin{solution}
	$2324.6_{10} = 100100010100.1001\overline{1001}_2$
\end{solution}
\end{frame}


