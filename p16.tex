\section{Pregunta N$^{\circ}$16a\qquad Khalid Izquierdo Ayllon}

\begin{frame}
	\begin{enumerate}\setcounter{enumi}{15}
		\item

		      Escribir en base dos los siguientes números dados en base $10$.


		      \begin{enumerate}[a)]
			      \item
                    
                    \begin{center}
                    \begin{align*}
                        \text{Parte entera: } & \\
                        2324_{10} \div 2 &= 1162 \text{ resto } 0 \\
                        1162 \div 2 &= 581 \text{ resto } 0 \\
                        581 \div 2 &= 290 \text{ resto } 1 \\
                        290 \div 2 &= 145 \text{ resto } 0 \\
                        145 \div 2 &= 72 \text{ resto } 1 \\
                        72 \div 2 &= 36 \text{ resto } 0 \\
                        36 \div 2 &= 18 \text{ resto } 0 \\
                        18 \div 2 &= 9 \text{ resto } 0 \\
                        9 \div 2 &= 4 \text{ resto } 1 \\
                        4 \div 2 &= 2 \text{ resto } 0 \\
                        2 \div 2 &= 1 \text{ resto } 0 \\
                        1 \div 2 &= 0 \text{ resto } 1 \\
                        \text{Para obtener la parte entera en base 2, leemos los restos de abajo hacia arriba: } & 100100010100
                    \end{align*}
                    \end{center}
                    
                \end{enumerate}
        \end{enumerate}
\end{frame}

\begin{frame}{Parte Fraccionaria}
                    \begin{align*}
                        \text{Parte fraccionaria: } & \\
                        \text{Multiplicamos la parte fraccionaria por 2: } & 0.6 \times 2 = 1.2 \\
                        \text{La parte entera es 1, por lo que escribimos un 1 a la derecha del punto: } & 0.1 \\
                        \text{Multiplicamos la parte fraccionaria por 2 de nuevo: } & 0.2 \times 2 = 0.4 \\
                        \text{La parte entera es 0, por lo que escribimos un 0 a la derecha del punto: } & 0.10 \\
                        \text{Multiplicamos la parte fraccionaria por 2 de nuevo: } & 0.4 \times 2 = 0.8 \\
                        \text{La parte entera es 0, por lo que escribimos un 0 a la derecha del punto: } & 0.100 \\
                        \text{Multiplicamos la parte fraccionaria por 2 de nuevo: } & 0.8 \times 2 = 1.6 \\
                        \text{La parte entera es 1, por lo que escribimos un 1 a la derecha del punto: } & 0.1001 \\
                        \text{Multiplicamos la parte fraccionaria por 2 de nuevo: } & 0.6 \times 2 = 1
                    \end{align*}

\begin{solution}
	$2324.6_{10} = 100100010100.1001\overline{1001}_2$
\end{solution}
\end{frame}


